%! suppress = MissingLabel
%! Author = David
%! Date = 26.05.2022

% Preamble
\documentclass[11pt]{article}

% Packages
\usepackage[ngerman]{babel}
\usepackage{color}
\usepackage[a4paper,lmargin={4cm},rmargin={2cm},
    tmargin={2.5cm},bmargin = {2.5cm}]{geometry}
\usepackage{amssymb}
\usepackage{amsthm}
\usepackage{graphicx}


\begin{titlepage}

    \title{Bidirektionales Laden}
    \author{David Behres}
    \date{2022}
    \maketitle
\end{titlepage}

% Document
\begin{document}

    \section{Über die Probleme der Stromversorgung}

    Sowohl der Klimawandel als auch der internationale Konflikt zwischen der Ukraine und Russland
    zeigen die Probleme der deutschen Stromversorgung auf.
    Deutschlands Energieerzeugung erfolgt zu 52,1\% aus konventionellen oder fossilen Energiequellen, die der Umwelt
    schaden.
    Die Abhängigkeit von konventionellen Energien treten in der Regel wegen Mangel an erneuerbaren Energien auf.
    Weitere Probleme sind der Import von Strom bei Produktionsmangel und Export von Strom bei Produktionsüberschuss.
    Zudem reicht die interne Energieproduktion manchmal, besonders bei windstillen Nächten, nicht aus, um den
    Energieverbrauch abzudecken, wodurch Strom importiert werden muss.
    Umgekehrt kommt es bei hoher Sonnen- und Windstärke oft zu Energieüberschuss.
    Da es nicht genügend Kapazitäten gibt, um diese überschüssige Energie zu speichern, muss sie exportiert werden.




    \bibliography{main}
    \bibliographystyle{plain}

\end{document}
