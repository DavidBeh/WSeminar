%! suppress = MissingLabel
%! Author = David
%! Date = 26.05.2022

% Preamble
\documentclass[12pt]{article}

% Zitieren

\usepackage{biblatex-chicago}
\addbibresource{main.bib}
% Packages

\usepackage[ngerman]{babel}
\usepackage{color}
\usepackage{amssymb}
\usepackage{amsthm}
\usepackage{graphicx}

\usepackage[a4paper,margin=3cm]{geometry}
\usepackage[onehalfspacing]{setspace}

\usepackage[pdftex,pdfpagelabels,bookmarks,hyperindex,hyperfigures]{hyperref}
\usepackage{biblatex}

\begin{titlepage}


    \title{Bidirektionales Laden}
    \author{David Behres}
    \date{2022}
    \maketitle

    TODO: Deckblatt nach Vorlage gestalten
\end{titlepage}


% Document
\begin{document}

    \tableofcontents


    \section{Über die Probleme der Stromversorgung}
        Sowohl der Klimawandel als auch der internationale Konflikt zwischen der Ukraine und Russland
        zeigen die Probleme der deutschen Stromversorgung auf.
        Deutschlands Energieerzeugung erfolgt zu 52,1\% aus konventionellen oder fossilen Energiequellen, die der Umwelt
        schaden.
        Die Abhängigkeit von konventionellen Energien treten in der Regel wegen Mangel an erneuerbaren Energien auf.
        Weitere Probleme sind der Import von Strom bei Produktionsmangel und Export von Strom bei Produktionsüberschuss.
        %Zudem reicht die interne Energieproduktion manchmal, besonders bei windstillen Nächten, nicht aus, um den
        %Energieverbrauch abzudecken, wodurch Strom importiert werden muss.
        %Umgekehrt kommt es bei hoher Sonnen- und Windstärke oft zu Energieüberschuss.
        %Da es nicht genügend Kapazitäten gibt, um diese überschüssige Energie zu speichern, muss sie exportiert werden.

        \subsection{Abhängigkeit von russischem Erdgas}
            Russland führt derzeit einen grausamen und unnötigen Überfall auf die Ukraine aus.
            Um Russland zu schwächen haben viele Staaten Sanktionen verhängt, welche das Land unter anderem
            Wirtschaftlich
            schwächen sollen.\footcite{sanktion}
            Dennoch bezieht Deutschland weiterhin hohe Mengen and Öl und Erdgas aus Russland, da es von diesen
            Energieträgern abhängig ist, und unterstützt damit indirekt den Krieg.
            Die Abhängigkeit von Erdgas ist ein wichtiger Faktor, der die Energieversorgung auf Deutschland

\end{document}

