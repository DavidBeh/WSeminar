%! suppress = MissingLabel
%! Author = David
%! Date = 26.05.2022

% Preamble
\documentclass[12pt, ngerman]{article}

% Zitieren
%\usepackage{biblatex-chicago}
\usepackage[backend=biber]{biblatex-chicago}
\addbibresource{main2.bib}
\usepackage{biblatex}
\interfootnotelinepenalty=10000
% Packages
%test

\usepackage[ngerman]{babel}

%\usepackage{color}
%\usepackage{amssymb}
\usepackage{amsthm}
\usepackage{graphicx}
\usepackage{pgf-pie}
\usepackage{pgfplots}
%\usepackage[section]{placeins}


\usepackage[a4paper,margin=3cm]{geometry}
\usepackage[onehalfspacing]{setspace}

\usepackage[pdftex,pdfpagelabels,bookmarks,hyperindex,hyperfigures]{hyperref}
\usepackage[nameinlink]{cleveref}
\usepackage{tikz}
\usepackage{siunitx}
\usepackage{booktabs}
\sisetup{locale = DE}

\newcommand{\pe}{_{\mathrm{PE}}}
\newcommand{\el}{_{\mathrm{el}}}

% Document
\begin{document}

\begin{titlepage}


    \title{Bidirektionales Laden}
    \author{David Behres}
    \date{2022}
    \maketitle

\end{titlepage}

\tableofcontents
\pagebreak


\section{Nutzen des Bidirektionalen Ladens für das deutsche Stromnetz}



Sowohl der Klimawandel als auch der internationale Konflikt zwischen der Ukraine und Russland
zeigen die Probleme der deutschen Stromversorgung auf.
Deutschlands Energieerzeugung erfolgt zu 52,1\% aus konventionellen oder fossilen Energiequellen, die der Umwelt
schaden.
Die Abhängigkeit von konventionellen Energien treten in der Regel wegen Mangel an erneuerbaren Energien auf.
Weitere Probleme sind der Import von Strom bei Produktionsmangel und Export von Strom bei Produktionsüberschuss.
%Zudem reicht die interne Energieproduktion manchmal, besonders bei windstillen Nächten, nicht aus, um den
%Energieverbrauch abzudecken, wodurch Strom importiert werden muss.
%Umgekehrt kommt es bei hoher Sonnen- und Windstärke oft zu Energieüberschuss.
%Da es nicht genügend Kapazitäten gibt, um diese überschüssige Energie zu speichern, muss sie exportiert werden.

\subsection{Abhängigkeit von russischem Gas und Öl} \label{subsec:putin}
Russland führt derzeit einen grausamen und unnötigen Überfall auf die Ukraine aus.
Um Russland zu schwächen haben viele Staaten Sanktionen verhängt, welche das Land unter anderem
Wirtschaftlich schwächen sollen.\footcite{SanktionenGegenRussland2022}
Dennoch bezieht Deutschland weiterhin hohe Mengen and Öl und Erdgas aus Russland, da es von diesen und
unterstützt damit indirekt den Krieg.



Im vergangenen Jahr bezog Deutschland 55\% des Erdgas aus Russland, gegen Ende dieses Jahres will man dies auf
30\% verringern.\footcite{wdraktuellFAQWasGasLieferstopp} Dennoch wird das nicht reichen um Russland merkbar zu
Schwächen, denn durch die vom Krieg verursachten hohen Gaspreise wird Deutschland nach Schätzungen der
Umweltorganisation Greenpeace Rekordsummen an Russland für Erdgas und Öl und zahlen und dadurch den russischen
Krieg finanzieren.\footcite{balserOelUndGas}

\subsection{Abhängigkeit von fossilen Brennstoffen und deren Einfluss auf den Klimawandel}

Der Anteil an konventionellen Energieträgern an der Stromerzeugung lag im 1.\ Quartal 2022 bei 52,1\%.
Die Verwendung solcher fossiler Brennstoffe, wie Braunkohle oder Erdgas, hat durch die Erzeugung von
Treibhausgasen einen schädlichen Einfluss auf den Klimawandel.\footcite{wilkeErneuerbareUndKonventionelle2013}

\subsection{Energieträger und ihre Umweltauswirkungen}

\subsubsection{Nutzung von Primärenergieträgern}\label{subsec:primar}

Der Energieverbrauch in Deutschland wird durch unterschiedliche Energieträger gedeckt.
Diese Energieträger unterscheiden sich unter anderem in ihrem Umwelteinfluss.

Das folgende Diagramm unterteilt den Primärenergieverbrauch des Jahres 2020 von
\qty{11.895}{PJ\pe}\footnote{PE steht für Primärenergie}
in die unterschiedlichen Energieträger.
Der Primärenergieverbrauch bezeichnet den Energiegehalt der verwendeten Energieträger und
hängt unter anderem vom Wirkungsgrad ab.

\begin{figure}[!h]
    \begin{tikzpicture}
        \pie[text=legend, sum=auto, radius=2, color={blue, yellow, gray, brown, lightgray, green, teal}]{
            4087/Öl,
            3144/Gas,
            896/Steinkohle,
            958/Braunkohle,
            702/Kernenergien,
            1972/Erneuerbare Energien,
            136/Sonstige Energieträger}
    \end{tikzpicture}\caption{Primärenergieverbrauch nach Energieträger 2020 in
        \unit{PJ\pe}}
\end{figure}


Davon werden etwa \qty{1257}{PJ\pe} Gas und \qty{807}{PJ\pe} Öl direkt zum Beheizen für den Gewerbe-, Handels-,
Dienstleistungs- und Wohnsektor, sowie \qty{2082}{PJ\pe} Öl als Brennstoff im Verkehrssektor verwendet.\footcite{Energieflussbild2020PJLang}
Durch Fernwärme werden ungefähr \qty{154}{PJ\pe} Gas und \qty{97}{PJ\pe} Steinkohle verbraucht.\footcite{WieKannTransformation}
Das Beheizen im Industriesektor durch Erdgas kann grob auf \qty{70}{PJ\pe} geschätzt werden.\footcite{Industrie}

\begin{table}[!h]
    \centering
    \begin{tabular}{r|llll}
        \toprule
                & Gas  & Öl   & Steinkohle & Braunkohle \\
        \midrule
        Heizen  & 1327 & 807  & 97         &            \\
        Verkehr &      & 2082 &            &            \\
        \bottomrule
    \end{tabular}
    \caption{Nutzung einiger Energieträger in \unit{PJ\pe}}
\end{table}

\subsubsection{Elektrische Energie}

Vom Stromverbrauch des Jahres 2021 von \qty{1810}{PJ\el} wurden \qty{344}{PJ\el} mit Braunkohle, \qty{181}{PJ\el} mit Gas
und \qty{181}{PJ\el} Steinkohle mit produziert.\footcite{SMARDEntwicklungenIm}.
Allerdings ist es wichtig diese Menge nicht mit den anderen zu vergleichen, da es sich
um die Erzeugnisse in elektrischer Energie handelt.

Energien müssen je nach Nutzen in eine bestimmte Energieform gebracht werden.
Dazu werden die an die Energieträger gebundenen Energien mit chemischen und technischen
Prozessen in die gewünschte Form gebracht.
Beispielsweise kann die an Brennstoffen gebundene chemische Energie
in thermische umgewandelt werden, wobei Treibhausgase freigesetzt werden.
Diese kann anschließend zum Heizen oder in der Industrie verwendet werden
aber auch in Kraftwerken weiter zu elektrischer Energie verarbeitet werden.
Dabei kommt allerdings, je nach Wirkungsgrad des Kraftwerks,
ein großer Teil der Energie als Wärmeenergie abhanden.\footcite{AachenHatEnergie}

Um später die Emissionen zu berechnen, müssen die elektrischen Energiemengen zuerst in
Primärenergie umgewandelt werden

Der Wirkungsgrad \unit{\eta} ist allgemein definiert als
\(\eta=\frac{E_{\mathrm{nutz}}}{E_{\mathrm{ges}}}\).\footcite{WirkungsgradLEIFIphysik}
Hier entspricht \(E\mathrm{_{nutz}}\) \(E\el\) und \(E_\mathrm{{ges}}\) \(E\pe\). Durch Umformen ergibt sich:

\[E\pe=\frac{E\el}{\eta}\]

Aus den durchschnittlichen Wirkungsgraden den jeweiligen Energieträgern und ihren Beiträgen
zur Erzeugung können die Primärenergien berechnet werden.

\begin{table}[!h]
    \begin{tabular}{r|llll}
        \toprule
                                           & Gas  & Öl & Steinkohle & Braunkohle \\
        \midrule
        Erzeugung in \unit{PJ\el}          & 181  &    & 181        & 344        \\
        Wirkungsgrad \protect\footnotemark & 0,49 &    & 0,44       & 0,39       \\
        \midrule
        Primärenergie in \unit{PJ\pe}      & 369  &    & 411        & 822        \\
        \bottomrule
    \end{tabular}
    \caption{Beteiligung fossiler Energieträger an Stromerzeugung 2021}
\end{table}
\footnotetext{\cite{wilkeKraftwerkeKonventionelleUnd2013}}

Der niedrige zeitliche Abstand der hier verwendeten Daten (2021) zu den in \cref{subsec:primar} (2020)
wird in den nächsten Abschnitten ignoriert.

\subsubsection{Ursache und Lösung der von konventionelle Energieträgern verursachten Klimaschäden}

Diese in der Tabelle aufgelisteten Energiemengen entsprechen etwa der Hälfte
der durch konventionelle Energieträger erzeugten Energien (Siehe \cref{fig:kon}).
Die Emissionen von Treibhausgasen, die durch den Verbrauch von konventionellen Energieträgern
entstehen, sind eine der größten Antreiber des Klimawandels.
Um den Klimawandel zu stoppen, ist es unabdingbar, Emissionen zu minimieren.
Eine Möglichkeit dies zu realisieren, neben des Reduzieren des Verbrauchs,
ist das Wechseln auf emissionsärmere, und damit klimafreundlichere,
Energieproduktionsmethoden. % Muss überarbietet werden

\begin{figure}[!h]
    \begin{tikzpicture}
        \pie[text=legend, sum=auto, color={yellow!60, blue!60, brown!60, red, green, teal}, radius=2]{
            2555/Heizen (fossil\; zu ers.),
            2082/Kraftstoff (fossi\; zu ers.),
            1602/Stromerzeugung (fossil\; zu ers.),
            3548/Sonstige fossilen,
            1972/Erneuerbare,
            136/Sonstige
        }
    \end{tikzpicture}
    \caption{Zu ersetzenden fossile Energieträger in \unit{PJ\pe}}
    \label{fig:kon}
\end{figure}

Als Alternative zum Heizen mit fossilen Brennstoffen bietet sich
das Heizen mit Wärmepumpen an, welche etwa \unit{\frac{2}{3}} der
Energie aus der Umgebung entziehen.

Als Alternative zum Verbrauch von fossilen Kraftstoffen im Verkehr bietet
sich das Fahren von Elektroautos an. Der Kraftstoffverbrauch macht einen
großen Teil des Gesamtenergieverbrauchs aus, und wirkt sich damit sehr
negativ auf den Klimawandel aus.

Damit beide oben genannten Lösungsansätze umgesetzt werden können, müssen
höhere Stromkapazitäten vorhanden sein. Diese müssten außerdem durch erneuerbare
Energien abgedeckt werden. Auch die bereits von konventionellen Energieträgern
abgedeckten Stromkapazitäten werden durch regenerative Abgedeckt werden müssen.

\subsubsection{Berechnung der Emissionen}\label{subsec:emission}

Die Menge der bei Verbrennung freigesetzten Treibhausgase varriert je
nach Energieträger. Um die Emissionen der Energieträger einzuschätzen, muss man diese mit dem Emissionsfaktor multiplizieren.
Der Emissionsfaktor gibt die Menge der Treibhausgasemissionen (CO\textsubscript{2} Äquivalent) an,
die bei der Produktion von Energie verbraucht wird.
Besonders bei der Stromerzeugung ist dieser wegen der niedrigen Effizienz sehr hoch.



\begin{table}[!htbp]
    \centering
    \begin{tabular}{r|llll}

        \toprule
         & Gas & Öl & Steinkohle & Braunkohle\\
        \midrule
        Emissionsfaktor & & & & \\
        \midrule
        Emissionen durch Heizen & & & & \\
        Emissionen durch Treibstoff & & & & \\
        Emissionen durch Stromerzeugung & & & & \\
        \bottomrule
    \end{tabular}
    \caption{Emissionsfaktoren der Energieträger je nach Verwendung\protect\footnotemark}
\end{table}
\footnotetext{\cite{quaschningSpecificCarbonDioxide}}



\subsection{Mangelnde Kapazitäten des Stromnetz um Heizträger zu ersetzen}
Bei direkter Betrachtung des Anteils von Erdgas bei der Stromerzeugung von etwa 10\% scheint Erdgas zunächst
kein großer Faktor in der elektrischen Stromversorgung zu sein.\footcite{SMARDHoherEEAnteil,EnergieWofuerErdgas}
Betrachtet man jedoch, dass Erdgas über die Hälfte des in Deutschland verbrauchten
Energie ausmacht und hauptsächlich zum Heizen verwendet wird\footcite{Anwendungsbereiche,EnergieWofuerErdgas},
während nur zu 5,6\% mit elektrischer Energie geheizt wird, erkennt man schnell, dass Deutschland sehr von diesen
fossilen Brennstoffen (aus Russland) abhängig ist.
Dies liegt daran, dass das Deutsche Stromnetz nicht die nötigen Kapazitäten hat, um diese zu
ersetzen.\footcite{EnergieWofuerErdgas}



\subsection{Unzuverlässigkeit der Erneuerbaren}
Dass die erweiterte Nutzung der erneuerbaren Energien unabdingbar für die Lösung des deutschen Energieproblemes
ist, sollte nun belegt sein. Allerdings sind die darunter relevanten Energiequellen, Wind und Sonne, nicht
zuverlässig, da sie von variierenden Umweltfaktoren abhängen. Zu schlechten Zeiten würde selbst bei einem Ausbau
von diesen die Stromversorgung auf konventionelle Energien und Importe zurückfallen. Um das zu vermeiden braucht
es hohe Speicherkapazitäten.

\subsection{Produktionsüberschuss bei Fotovoltaik-Ausbau}
Bei Nutzung der verfügbaren Fläche für Fotovoltaikanlagen würde es bei dem derzeitigen Stromnetz, Verbrauch und
Speicherkapazitäten ohne Zweifel zu einem Produktionsüberschuss kommen.\footcite{wirthAktuelleFaktenZur}
Der Ausbau wäre also bis zu einem bestimmten Punkt nicht mehr sinnvoll, da es an Speicherkapazitäten fehlt.
Dieses Problem könnte durch Bidirektionales Laden gelöst werden.


\section{Bidirektionales Laden als Lösung}

\subsection{Vehicle to Home}

\subsection{Vehicle to Grid \& Einspeisung}

\subsection{Auswirkungen auf Batterielebensdauer}

\subsection{Aktueller Fortschritt}


\section{Experiment: 12V Batterie}

\subsection{Versuchsaufbau}

\end{document}

