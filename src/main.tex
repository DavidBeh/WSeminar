%! suppress = MissingLabel
%! Author = David
%! Date = 26.05.2022

% Preamble
\documentclass[12pt]{article}

% Zitieren
%\usepackage{biblatex-chicago}
\usepackage[backend=biber]{biblatex-chicago}
\addbibresource{main2.bib}
\usepackage{biblatex}
\interfootnotelinepenalty=10000
% Packages

\usepackage[ngerman]{babel}

%\usepackage{color}
%\usepackage{amssymb}
\usepackage{amsthm}
\usepackage{graphicx}

\usepackage[a4paper,margin=3cm]{geometry}
\usepackage[onehalfspacing]{setspace}

\usepackage[pdftex,pdfpagelabels,bookmarks,hyperindex,hyperfigures]{hyperref}



\begin{titlepage}


    \title{Bidirektionales Laden}
    \author{David Behres}
    \date{2022}
    \maketitle

    TODO: Deckblatt nach Vorlage gestalten
\end{titlepage}


% Document
\begin{document}

    \tableofcontents
    \pagebreak


    \section{Nutzen des Bidirektionalen Ladens für das deutsche Stromnetz}
    Sowohl der Klimawandel als auch der internationale Konflikt zwischen der Ukraine und Russland
    zeigen die Probleme der deutschen Stromversorgung auf.
    Deutschlands Energieerzeugung erfolgt zu 52,1\% aus konventionellen oder fossilen Energiequellen, die der Umwelt
    schaden.
    Die Abhängigkeit von konventionellen Energien treten in der Regel wegen Mangel an erneuerbaren Energien auf.
    Weitere Probleme sind der Import von Strom bei Produktionsmangel und Export von Strom bei Produktionsüberschuss.
    %Zudem reicht die interne Energieproduktion manchmal, besonders bei windstillen Nächten, nicht aus, um den
    %Energieverbrauch abzudecken, wodurch Strom importiert werden muss.
    %Umgekehrt kommt es bei hoher Sonnen- und Windstärke oft zu Energieüberschuss.
    %Da es nicht genügend Kapazitäten gibt, um diese überschüssige Energie zu speichern, muss sie exportiert werden.

    \subsection{Abhängigkeit von russischem Gas und Öl} \label{subsec:putin}
    Russland führt derzeit einen grausamen und unnötigen Überfall auf die Ukraine aus.
    Um Russland zu schwächen haben viele Staaten Sanktionen verhängt, welche das Land unter anderem
    Wirtschaftlich schwächen sollen.\footcite{SanktionenGegenRussland2022}
    Dennoch bezieht Deutschland weiterhin hohe Mengen and Öl und Erdgas aus Russland, da es von diesen und
    unterstützt damit indirekt den Krieg.

    Im vergangenen Jahr bezog Deutschland 55\% des Erdgas aus Russland, gegen Ende dieses Jahres will man dies auf
    30\% verringern.\footcite{wdraktuellFAQWasGasLieferstopp} Dennoch wird das nicht reichen um Russland merkbar zu
    Schwächen, denn durch die vom Krieg verursachten hohen Gaspreise wird Deutschland nach Schätzungen der
    Umweltorganisation Greenpeace Rekordsummen an Russland für Erdgas und Öl und zahlen und dadurch den russischen
    Krieg finanzieren.\footcite{balserOlUndGas}



    \subsection{Abhängigkeit von fossilen Brennstoffen und deren Einfluss auf den Klimawandel}

    Der Anteil an konventionellen Energieträgern an der Stromerzeugung lag im 1.\ Quartal 2022 bei 52,1\%.
    Die verwendung solcher fossiler Brennstoffe, wie Braunkohle oder Erdgas, hat durch die Erzeugung von
    Treibhausgasen einen Schädlichen einfluss auf den Klimawandel.\footcite{wilkeErneuerbareUndKonventionelle2013}
    
    \subsection{Mangelnde Kapazitäten des Stromnetz um Heizträger zu ersetzen}
    Bei direkter Betrachtung des Anteils von Erdgas bei der Stromerzeugung von etwa 10\% scheint Erdgas zunächst
    kein großer Faktor in der elektrischen Stromversorgung zu sein.\footcite{SMARDHoherEEAnteil, EnergieWofurErdgas}
    Betrachtet man jedoch, dass Erdgas über die Hälfte des in Deutschland verbrauchten
    Energie ausmacht und hauptsächlich zum Heizen verwendet wird\footcite{Anwendungsbereiche, EnergieWofurErdgas},
    während nur zu 5,6\% mit elektrischer Energie geheizt wird, erkennt man schnell, dass Deutschland sehr von diesen
    fossilen Brennstoffen (aus Russland) abhängig ist.
    Dies liegt daran, dass das Deutsche Stromnetz nicht die nötigen Kapazitäten hat, um diese zu
    ersetzen.\footcite{EnergieWofurErdgas}
    
    \subsection{Unzuverlässigkeit der Erneuerbaren}
    Dass die erweiterte Nutzung der Erneuerbaren Energien unabdingbar für die Lösung des deutschen Energieproblemes
    ist, sollte nun belegt sein. Allerdings sind die darunter relevanten Energiequellen, Wind und Sonne, nicht
    zuverlässig, da sie von variierenden Umweltfaktoren abhängen. Zu schlechten Zeiten würde selbst bei einem Ausbau
    von diesen die Stromversorgung auf konventionelle Energien und Importe zurückfallen. Um das zu vermeiden braucht
    es hohe Speicherkapazitäten.

    \subsection{Produktionsüberschuss bei Fotovoltaik-Ausbau}
    Bei Nutzung der verfügbaren Fläche für Fotovoltaikanlagen würde es bei dem derzeitigen Stromnetz, Verbrauch und
    Speicherkapazitäten ohne Zweifel zu einem Produktionsüberschuss kommen.\footcite{wirthAktuelleFaktenZur}
    Der Ausbau wäre also bis zu einem bestimmten Punkt nicht mehr sinnvoll, da es an Speicherkapazitäten fehlt.
    Dieses Problem könnte durch Bidirektionales Laden gelöst werden.

    \section{Bidirektionales Laden als Lösung}

    \subsection{Vehicle to Home}

    \subsection{Vehicle to Grid \& Einspeisung}

    \subsection{Auswirkungen auf Batterielebensdauer}

    \subsection{Aktueller Fortschritt}


    \section{Experiment: 12V Batterie}

    \subsection{Versuchsaufbau}

\end{document}

